\documentclass[class=article, crop=false]{standalone}
\usepackage[utf8]{inputenc} % allow utf-8 input
\usepackage[T1]{fontenc}    % use 8-bit T1 fonts
\usepackage{url}            % simple URL typesetting
\usepackage{booktabs}       % professional-quality tables
\usepackage{amsfonts}       % blackboard math symbols
\usepackage{nicefrac}       % compact symbols for 1/2, etc.
\usepackage{microtype}      % microtypography
\usepackage{lipsum}
\usepackage{amsmath}
\usepackage{amsthm}
\usepackage{hyperref}
\usepackage{import}
\usepackage[subpreambles=true]{standalone}
\hypersetup{
    colorlinks=true, %set true if you want colored links
    linktoc=all,     %set to all if you want both sections and subsections linked
    linkcolor=blue,  %choose some color if you want links to stand out
}

\theoremstyle{definition}
\newtheorem{definition}{Definition}[section]

\theoremstyle{remark}
\newtheorem*{remark}{Remark}

\theoremstyle{lemma}
\newtheorem*{lemma}{Lemma}

\theoremstyle{theorem}
\newtheorem*{theorem}{Theorem}

\theoremstyle{corollary}
\newtheorem*{corollary}{Corollary}

\theoremstyle{property}
\newtheorem*{property}{Property}
\usepackage[subpreambles=true]{standalone}
\usepackage{import}
\begin{document}

\section{Directed Acyclic Graphs and Do-Calculus}
	We follow the formalism of structural causal models (SCMs). 


	The following proposition summarizes Pearl's do-calculus rules:

	First, there are two basic criterion that one can take advantage of when decomposing intervention quantities (i.e. P(Y|do(x)))  into observable quantities (i.e. PY(y|x)).

	\subsection{Back-door criterion}
		Suppose we are given a causal diagram, G, with observed data on a subset V of observed variables in G, and we \textbf{wish to estimate what effect the intervention do(X=x) has on response variables Y}. I.e. What is $P(y|\hat{x})$.

		The back-door criterion is a graphical test that can be applied to the causal diagram to test if a set $Z \subset V$ of variables is sufficient for identifying $P(y | \hat{x})$.

		\begin{definition} [Back-Door Criterion]
			A set of variables Z, satisfies the back-door criterion relative to an ordered pair of variables $(X_i, X_j)$ in a DAG G, if:

			i) no node in Z is a descendent of $X_i$ and

			ii) Z blocks every path between $X_i$ and $X_j$ that contains an arrow into $X_i$.

			If X and Y are disjoint subsets of nodes in G, then Z satisfies the back-door criterion relative to $(X,Y)$ such that it satisfies the criterion relative to any pair $(X_i, X_j)$ for $X_i \in X$ and $X_j \in Y$.
		\end{definition}

	\subsection{Front-door criterion}
		Compared to the back-door criterion, the \textbf{front-door criterion} is a method for identifying causal effects by measuring mediators of the causal effect. 

		\begin{definition} [Front-Door Criterion]
			A set of variables, M, satisfies the front-door criterion when:

			i) M blocks all directed paths from X to Y and 

			ii) there are no unblocked back-door paths from X to M and 

			iii) X blocks all back-door paths from M to Y.

			Condition i) implies that all effects of X on Y occur by \textbf{mediation} through M!

			Condition ii) says that we can obtain the effect of X on M directly because there are no unobserved confounders affecting both.

			Condition iii) specifies that X satisfies the back-door criterion for identifying the effect of M on Y.
		\end{definition}

	\subsection{d-separation and conditional independence}
		In Bayesian networks, the key assumption is that every variable is conditionally independent of its non-descendents, given its parents. This notion can be formalized in the context of \textbf{d-separation}.

		Given a causal graph, G, we say that X is d-separate from Y given Z if: If there are no collider-free paths between X and Y that does not traverse Z. If there is a collider-free path between X and Y that does not traverse Z, then we say X and Y are d-connected.

		This is useful for writing conditional independence statements, such as:

			$$P(X | Y, Z) = P(X | Z)$$

		if X and Y are d-separated by Z. 

	\subsection{c-separation and c-connectedness}
		In Tian, 2002, a fundamental condition for identifying causal effects that is more general then the back-door and front-door criterion was developed.

	\subsection{References}
		J. Tian and J. Pearl. 2002. A General Identification Condition for Causal Effects. In Proceedings of the Eighteenth National Conference on Artificial Intelligence (AAAI 2002), pp. 567–573. AAAI Press/The MIT Press, Menlo Park, CA.
\end{document}